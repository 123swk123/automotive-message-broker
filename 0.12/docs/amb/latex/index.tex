\hypertarget{index_intro}{}\section{Introduction}\label{index_intro}
A\-M\-B Library documentation outlines the internal classes and structures for building plugins for A\-M\-B.\hypertarget{index_architecture}{}\section{General Architecture}\label{index_architecture}
A\-M\-B has 3 main parts. Source plugins which provide data, a routing engine that routes data and sink plugins that consume the data.\hypertarget{index_plugins}{}\section{Plugins}\label{index_plugins}
There are two types of plugins\-: plugins that provide data, called \char`\"{}sources\char`\"{} (\hyperlink{classAbstractSource}{Abstract\-Source}) and plugins that consume data, called \char`\"{}sinks\char`\"{} (\hyperlink{classAbstractSink}{Abstract\-Sink}). A typical source would get data from the vehicle and then translate the raw data into A\-M\-B property types. Sinks then subscribe to the property types and do useful things with the data.

Example plugins can be found in plugins/exampleplugin.\{h,cpp\} for an example source plugin and plugins/examplesink.\{h,cpp\} for an example sink plugin. There are also many different types of plugins useful for testing and development in the plugins/ directory.\hypertarget{index_routing_engine}{}\section{Routing Engine Plugins}\label{index_routing_engine}
As of 0.\-12, the routing engine itself can be exchanged for a plugin. This allows users to swap in routing engines with different behaviors, additional security, and custom throttling and filtering features.

The easiest way to get started creating a routing engine plugin would be to look at \hyperlink{classAbstractRoutingEngine}{Abstract\-Routing\-Engine}, the base class for all routing engines and the default routing engine in ambd/core.\-cpp. 